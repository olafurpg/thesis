
\section{Introduction} % (fold)
\label{sec:Introduction}
\lstset{style=scala}
The main motivation of this study is to bring scalafmt, a new Scala code formatter, to the Scala community.
The goal is to capture many popular coding styles so that a wide part of the Scala community can enjoy the benefits that come with automatic code formatting.

Without code formatters, software developers are responsible for manipulating all syntactic trivia in their programs.
What is syntactic trivia?
Consider the Scala code snippets in listings~\ref{lst:unformatted} and~\ref{lst:formatted1}.

\begin{minipage}{.45\textwidth}
\lstinputlisting[label={lst:unformatted}, caption=Unformatted code]{code/unformatted.scala}
\end{minipage}
\hfil
\begin{minipage}{.45\textwidth}
\lstinputlisting[label={lst:formatted1}, caption=Formatted code]{code/target/formatted1.scala}
\end{minipage}
% \hfil
Both snippets represent the same program.
The only difference lies in their syntactic trivia, that is where spaces and line breaks are used.
% The snippet in listing~\ref{lst:formatted1} is result of running \scalafmt{} on listing~\ref{lst:unformatted} with a constrait of fitting maximum 35 characters on each line.
Although the whitespace does not alter the execution of the program, listing~\ref{lst:formatted1} is arguably easier to read, understand and maintain for the software developer.
The promise of code formatters is to automatically convert any program that may contain style issues, such as in listing~\ref{lst:unformatted}, into a readable and consistent looking program, such as in listing~\ref{lst:formatted1}.
Automatic code formatting offers several benefits.
% Without a code formatters, the software developer is left to come up with the layout she prefers the most.
% With a code formatter, the choice is 
% In this example, good code formatting offers great value to the software developer.

Code formatting enables large-scale refactoring.
Google used ClangFormat\autocite{jasper_clang-format_????}, a code formatter, to migrate legacy C++ code to the modern C++11 standard\autocite{wright_large-scale_2013}.
ClangFormat was used to ensure that the refactored code adhered to Google's extensive C++ coding style\autocite{_google_????}.
Similar migrations can be expected in the near future for the Scala community once new dialects, such as Dotty\autocite{rompf_f_2015}, gain popularity.

Code formatting is valuable in collaborative coding environments.
The Scala.js project\autocite{_scala.js_????} has over 40 contributors and the Scala.js coding style\autocite{_scala.js_????-1} contains over 2.600 words.
Each contributor to the Scala.js project is expected to follow the coding style.
Each contributed patch is manually verified against the coding style by the project maintainers.
This adds a burden on both contributors and maintainers.
Several maintainers of popular Scala libraries have expressed this sentiment.
ENSIME\autocite{_ensime_????} is a popular Scala interaction mode for text editor.
Sam Halliday, a maintainer of ENSIME, says ``I don't have time to talk about formatting in code reviews. I want the machine to do it so I can focus on the design.''\autocite{halliday_i_2016-1}.
Akka\autocite{_akka_????} is another Scala library to build concurrent and distributed applications.
Viktor Klang, a maintainer of Akka, suggests a better alternative ``Code style should not be enforced by review, but by automate rewriting. Evolve the style using PRs against the rewriting config.''.\autocite{klang_code_2016}.
With code formatters, software develoeprs can direct their full focus on writing correct, maintainable and fast code.

\subsection{Whitespace style issues}
Mechanical style issues are source code issues that can be fixed automatically.
For example, consider listing~\ref{lst:mechanic}.
\lstinputlisting[label={lst:mechanic}, caption=Style issues]{code/mechanic.scala}
This code snippet leaves a lot left to be desired.
Mechanical issues include the redundant spaces around parentheses, unnecessary semicolon after the \texttt{println} and the inconsistent indentation in the body of the if statement.
Non-mechanical issues may include the fact the program prints an error message to the standard output and exits the process, instead of throwing an exception.

This thesis only addresses the whitespace style issues.
That is, to automatically fix the spaces and newline characters between the non-whitespace tokens in the original source code.
We will leave it to the software developer to decide whether the unnecessary semicolon should remain in the source file or if the program should throw an exception.
% Another example, the Spark style guide\autocite{spark-scala} follows the ``Rule of 30''\autocite{software_refactoring}.
% The rule of 30 dictates amongst several things that method bodies should not contain more than 30 lines of code.
% Although it is easy to statically detect a violation of that rule, it is non-trivial to automatically rewrite a method into smaller pieces of code.



% Code formatters have existed for a long time.
% In more recent years we've seen many new opimization based code formatters, such as clang-format, dartfmt and rfmt.
% Optimization based code formatters define a fit function for a given layout.
% A good layout has low cost while a bad layout has a high cost.
% The formatter runs an algorithm to choose the layout with the lowest code.
% This idea is not new, for example TeX uses a fit function to optimally break lines
% in text documents.

% Much work has been put into developing language agnostic code formatters or pretty printers.
% Researchers of such work correctly highlights that developing a custom code formatter for each language requires a lot of effort.
% It may be possible that we find a perfect way to develop language independent code formatters, see~\autocite{mps_article}.
% Still, today no such framework exists and people who write Scala code each day are still choosing to manually format their code.
% The goal of this thesis is not to address language agnostic code formatting.
% \subsection{Non-mechanical style issues}
% TODO(olafur)

% \subsection{Problem statement}
% This thesis addresses the problem of developing a code formatter, \scalafmt, for the Scala programming language.
% This formatter should:
% \begin{itemize}
%   \item capture Scala's most common idioms and popular coding styles, most importanly a line length limit.
%   \item be opinionated, it should be able to produce nice looking code even for the most egregiously formatted input.
%   \item be fast, <200ms for 99 percentile of source files, <2s for rest.
% \end{itemize}

\subsection{Contributions}
The main contribution presented in this thesis are the following:
\begin{itemize}
  \item scalafmt, a code formatter for the Scala programming language.
    At the time of this writing, scalafmt has been available for 3 months, it has been installed over 5.000 times and is already in use by several open source Scala libraries.
    For details on how to install and use scalafmt, refer to the scalafmt online documentation\autocite{geirsson_scalafmt_????}.
    % Discussion of scalafmt adoption is presented in section~\ref{sec:adoption}.

  \item algorithms and data structures to implement line wrapping under a maximum column-width limit.
    This work is presented in section~\ref{sec:algorithms}.
  \item tools to develop and test code formatters.
    This work is presented in section~\ref{sec:tooling}.
\end{itemize}
The scalafmt formatter itself may only be of direct interest to the Scala community.
However, much work in this thesis is not specifically tied to Scala and we hope can inspire the design of code formatters for other programming languages.

