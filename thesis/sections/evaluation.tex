\section{Evaluation}
Code formatting is inherently a subjective topic.
This introduces a challenge when evaluating a code formatter.
In this chapter, we will present measurements that we believe show the success of scalafmt.
We do not measure how well software developers perceive scalafmt formatted code.
Instead, we will focus on \emph{performance benchmarks} and \emph{user adoption}.

\subsection{Performance benchmarks}
In this chapter, we describe our test methodology, key metrics and the performance results.

\subsubsection{Test methodology}
We run micro benchmarks to get insight into how scalafmt performs in an interactive software developer workflow where scalafmt runs on file save.
We use the OpenJDK Java Microbenchmark Harness (JMH)\autocite{OpenJ38:online} to run micro benchmarks.
JMH can be used to build, run and analyze benchmarks written in languages targeting the JVM, including Scala.
The sbt-jmh\autocite{ktoso84:online} plugin makes it easy to integrate JMH with a Scala project.
For accurate measurements, we run 10 warmup iterations followed by 10 measured iterations.
We compare the running time with Scalariform, an alternative code formatter for Scala discussed in section~\ref{sec:scalariform}.
The micro benchmark is single threaded.

We run macro benchmarks to get insights into how scalafmt performs in a continuous integration setup.
We format a sample of 1.219.235 lines of code from 9.423 source files.
Figure~\ref{fig:macro_sample} shows the distribution of file sizes.
\begin{figure}\label{fig:macro_sample}
  \centering
  \input{target/eg2.tex}
  \caption{Distribution in file size.}
\end{figure}
The macro benchmark is multi-threaded.

Our hardware is a Macbook Pro (Retina, 15-inch, Mid 2014) laptop with a quad-core 2.5 GHz Intel Core i7 processor, 256 KB L2 cache per core and 6 MB shared L3 cache.
The laptop has 16 GB 1600 MHz DDR3 memory.
The operating system is OS X El Capitan 10.11.5.
We run the benchmarks from the scalafmt commit id \href{https://github.com/olafurpg/scalafmt/tree/aff5e794dae4787b08243f8abb87a3ca4d907e40}{aff5e794} compiled against Scala 2.11.7, running on
on JVM version 8, update 91.

\subsubsection{Performance metrics}
We are concerned with the time it takes to format a typical source file.
However, source files come in different shapes and sizes.
To accommodate this variety in files we select a small sample files from popular open source libraries.
The files will vary in sizes from small (<200 LOC), medium (~1.0000 LOC) and large (>4.000 LOC).
We measure the performance metric \texttt{ms/file}, how many milliseconds are required to format a source file.

\subsubsection{Results}
TODO.

\subsection{Adoption}\label{sec:adoption}
TODO


