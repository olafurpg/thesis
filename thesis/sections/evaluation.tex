\section{Evaluation}\label{sec:evaluation}
Code formatting is inherently a subjective topic.
This introduces a challenge when evaluating a code formatter.
In this chapter, we will present measurements that we believe show the success of scalafmt.
We do not measure how well software developers perceive scalafmt formatted code.
Instead, we will focus on \emph{performance benchmarks} and \emph{user adoption}.

\subsection{Performance benchmarks}
This chapter describes the 

\subsubsection{Setup}
The benchmarks are run on a Macbook Pro (Retina, 15-inch, Mid 2014) laptop with a quad-core 2.5 GHz Intel Core i7 processor, 256 KB L2 cache per core and 6 MB shared L3 cache.
The laptop has 16 GB 1600 MHz DDR3 memory.
The operating system is OS X El Capitan 10.11.5.
We run the benchmarks from the scalafmt commit id \href{https://github.com/olafurpg/scalafmt/tree/aff5e794dae4787b08243f8abb87a3ca4d907e40}{aff5e794} compiled against Scala 2.11.7, running on JVM version 8, update 91.
For accurate measurements, all benchmarks are run with the OpenJDK Java Microbenchmark Harness (JMH)\autocite{OpenJ38:online}.
JMH takes into account a variety of parameters that affect performance on the JVM.
The sbt-jmh\autocite{ktoso84:online} plugin makes it easy to integrate JMH with a Scala project.

\subsubsection{Macro benchmark}
The macro benchmark is designed to get insight on how scalafmt performs in a continuous integration setup.
For example, it is common to assert before code review that all source files are properly formatted.
For this benchmark we format the entire Scala.js codebase.
The codebase contains 915 source files and over 106 thousand lines of code, excluding blank lines and comments.
For accurate measurements, we run five iterations of the macro benchmark.
We compare the running time with Scalariform.

Table~\ref{tab:macro} shows the results from the macro benchmark.
\begin{table}
  \centering
  \caption{Results from macro benchmark.}\label{tab:macro}
  \begin{tabular}{lllll}
  Benchmark                &  Cores  &   Score  &   Error &  Units\\
  \hline
  \hline
  Parallel.scalafmt        &  4      &  14.616  &   0.632 &   s/op\\
  Parallel.scalariform     &  4      &   2.810  &   0.641 &   s/op\\
\hline
  Ratio &  & 5.20 &  & \\
  \\
  Synchronous.scalafmt     &  1      &  35.654  &   0.459 &   s/op\\
  Synchronous.scalariform  &  1      &   5.951  &   0.135 &   s/op\\
\hline
  Ratio &  & 5.99 &  & 
\end{tabular}

\end{table}
Scalafmt is almost 6x slower than Scalariform.
Why is the performance gap so big?
Is this gap acceptable for continuous integration setups?

We believe two factors contribute to the fact that scalafmt is 6x slower than Scalariform.
Firstly, scalafmt's formatting algorithm is more complex.
Scalafmt may try thousands of different formatting layouts to find an optimal formatting output.
In contrast, Scalariform's formatting algorithm is linear.
Secondly, scalafmt's performance is tied with scala.meta, which is still a young library with opportunities to become better optimized.
For example, profiling scalafmt reveals that 30\% of the formatting time is dedicated to a pre-processing step --- unrelated to the best-first search --- that could be accomplished with minimal overhead during parsing inside scala.meta.
In addition, preliminary results indicate that scalafmt may speed up yet another 30\% by upgrading to the latest experimental release of scala.meta\footnote{
scala.meta has gone through several difficult non-source-compatible upgrades during the development of scalafmt.}.

We believe the current performance is acceptable for continuous integration setups.
A typical diff in a code review touches only a few source files, and definitely far from 106 thousand lines of code as we test in this macro benchmark.
Moreover, if the build server has multiple CPU cores the performance is less of an issue.
However, we admit the current performance is far from excellent.
We have reason to believe there are plenty of opportunities to improve scalafmt's performance.

\subsubsection{Micro benchmark}
The micro benchmark is designed to get insight into how scalafmt performs in an interactive software developer workflow.
For example, it is common to configure a text editor to format source code on every save.
Before we run the benchmark, we must find out how many lines of code a typical source file contains.

We performed a small study to learn the size of a typical source file.
We collected a sample of 3.2 million lines of code from 33 open source Scala projects.
Table~\ref{tab:macro_sample} shows the distribution of file sizes in our sample.
\begin{table}
  \centering
  \caption{Percentiles of lines of code per file in micro benchmark.}~\label{tab:macro_sample}
  \begin{tabular}{llllllll}
  25th & Median & Mean & 75th & 90th & 95th & 99th & Max\\
  \hline
  \hline
  16 & 46 & 106 & 113 & 248 & 400 & 945 & 11.723\\
  
\end{tabular}

\end{table}
Observe that over 90 percent of all files are rather small, or under 250 lines of code.
Only one percent of files contain more than 1.000 lines of code.
Still, we assume developers spend quite a lot of time editing such large files.

Using the results from our small study, we choose to run the micro benchmark on four files of varying sizes: small ($\sim$ 50 LOC), medium ($\sim$300 LOC), large ($\sim$1.000 LOC) and extra large ($\sim$4.500 LOC).
To minimize error margins, we run 10 warmup iterations followed by 10 measured iterations.
As in the macro benchmark, we compare the running time with Scalariform.
The micro benchmark is single threaded.

Table~\ref{tab:micro} shows the results from the micro benchmark.
\begin{table}
  \centering
  \caption{Results from micro benchmark.}\label{tab:micro}
  \input{target/micro.tex}
\end{table}
No surprise, scalafmt is again slower than Scalariform.
Is this performance gap acceptable for interactive software development?

We believe this performance is usable for occasional code formatting, but not suitable for a workflow that formats on every save/compilation.
Amazon famously showed that sales decreased by 1 percent for every 100ms increase in page load time\autocite{kohavi2007online}.
We believe similar principles apply to scalafmt, every additional millisecond decreases the utility of scalafmt.
Still, we believe with scalafmt's formatting output is appealing enough to outweigh the slow performance.
As we'll discuss in the following section, our users seem to agree.

\subsection{Adoption}\label{sec:adoption}
Scalafmt has received quite some attention since its release in early March, three months ago.
In this section we present the statistics we believe demonstrate that scalafmt is --- despite its young age --- already proving itself useful for the Scala community.
All data points are as of June 9th, 2016.

\subsubsection{Installations}
Scalafmt has been installed over 6.500 times.
  Table~\ref{tab:installs} shows the installation numbers for each official distribution channel.
  \begin{figure}
    \CenterFloatBoxes
    \begin{floatrow}
      \ffigbox
      {\includegraphics[width=0.4\textwidth,angle=-90]{target/month.eps}}
      {\caption{Installations by month by channel.}\label{fig:installs}}
      \killfloatstyle
      \ttabbox{\input{target/installs.tex} }{
        \caption{Download numbers for scalafmt.}\label{tab:installs}}
    \end{floatrow}
  \end{figure}
IntelliJ is the Jetbrains plugin repository\footnote{
  See \url{https://plugins.jetbrains.com/plugin/8236?pr=}
}.
Data is not available for how many users built scalafmt from source.
The numbers represent absolute download numbers, not unique users.
Observe that v0.2.5 was released 22 days ago, meaning it has been installed 80 times on average per day since its release.
Considering the v0.2.5 installations numbers, it is fair to estimate that scalafmt currently has at least 1.000 users.

Figure~\ref{fig:installs} shows the growth in installations by month.
Observe that growth has doubled with each month.
Github represented proportionally many installs in the first month but only represents a small fraction of installation in May.
Maven installations quadrupled in May, taking the lead from the IntelliJ plugin in April.
We believe this increase in Maven installations is caused by projects installing the scalafmt SBT plugin for every test run in a continuous integration setup.


\subsubsection{Other}
We present interesting data points from a variety of disparate data sources:
\begin{table}
  \hyphenpenalty=1000
  \caption{Open source libraries that have reformatted their codebase with scalafmt and their customized settings.}\label{tab:oss}
  \begin{tabular}{lp{90mm}}
    Project & Customized coding style \\
    \hline \hline
    Scala Native\tablefootnote{
      \url{https://github.com/scala-native/scala-native}} & \texttt{defaultWithAlign} base style,
                                                            80 character column limit,
                                                            Java docstrings. \\
    Scala.js dom\tablefootnote{
      \url{https://github.com/scala-js/scala-js-dom}}     & \texttt{Scala.js} base style:
                                                            80 character column limit,
                                                            vertical alignment on case arrows,
                                                            bin packed arguments/parameters/parent constructors,
                                                            Java docstrings. \\
    Fetch\tablefootnote{
      \url{https://github.com/47deg/fetch}}               & \texttt{defaultWithAlign} base style,
                                                            100 character column limit,
                                                            Scala docstrings. \\
    psp-std\tablefootnote{
      \url{https://github.com/paulp/psp-std}}             & Customized vertical alignment,
                                                            160 character column limit,
                                                            2 space continuation indent,
                                                            spaces in import curly braces,
                                                            Scala docstrings.

  \end{tabular}
\end{table}
\begin{itemize}
  \item Several popular open-source libraries have reformatted their codebases with scalafmt.
    Table~\ref{tab:oss} shows an incomplete list libraries that have so far taken the jump.
    Observe that all libraries take advantage of vertical alignment.
    Moreover, each library customize on top of the base \texttt{default} style.
    It is worth mentioning that all libraries except Scala.js dom are relatively newly released.
    We believe more mature libraries wait longer before adopting such a new technology.
  \item The scalafmt code repository has received contributions from 8 external contributors.
    Several of these contributions added non-trivial features to scalafmt, including
    new configuration flags and extensions to the Router.
  \item 34 unique users, excluding the author, have opened a total of 138 tickets on the scalafmt issue tracker.
  \item The scalafmt Gitter\footnote{See \url{https://gitter.im/olafurpg/scalafmt}} instant messaging channel has 47 members. The channel is used to informally discuss bugs, new features and more.
  \item The user documentation website\footnote{See \url{http://scalafmt.org}} has been visited 5.422 times with an average visit duration of 98 seconds.
\end{itemize}



