\section{Evaluation}
Code formatting is inherently a subjective topic.
This introduces a challenge when evaluating a code formatter.
In this chapter, we will present measurements that we believe show the success of scalafmt.
We do not measure how well software developers perceive scalafmt formatted code.
Instead, we will focus on \emph{performance benchmarks} and \emph{user adoption}.

\subsection{Performance benchmarks}
In this chapter, we describe our test methodology, key metrics and the performance results.

\subsubsection{Test methodology}
We use JMH\autocite{OpenJ38:online} to benchmark scalafmt.
JMH stands for the Java Microbenchmark Harness and can be used to build, run and analyze benchmarks written in languages targeting the JVM.
The sbt-jmh\autocite{ktoso84:online} plugin makes it easy to integrate JMH with a Scala project.
Our hardware is a Macbook Pro (Retina, 15-inch, Mid 2014) with a 2.5 GHz Intel Core i7 processor and 16 GB 1600 MHz DDR3 memory.
The operating system is OS X El Capitan 10.11.5.
We run the benchmarks on the JVM version 8, update 91, and Scala 2.11.7.

\subsubsection{Performance metrics}
We benchmark against several parameters: lines of code,s
\subsubsection{Results}

\subsection{Adoption}\label{sec:adoption}

