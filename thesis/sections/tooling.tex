\section{Tooling}\label{sec:tooling}
This chapter describes the tools that we developed while designing an implementing algorithms for scalafmt.
These tools were indispensable in giving us confidence that our algorithms worked as intended.

\subsection{Heatmaps}
Section~\ref{sec:optimizations} introduces several extensions to algorithm~\ref{alg:bfsv1} that were required go get acceptable performance.
In general, the extensions involved eliminating search states.
To identify code patterns that triggered excessive search growth, we developed heatmaps.

Heatmaps are a visualization that displays which code regions are most frequently visited in the best-first search.
Figure~\ref{fig:heatmap} shows an example heatmap.
\begin{figure}
  \centering
  \includegraphics[width=\textwidth]{img/heatmap.png}
  \caption{Example heatmap with 5.121 visisted states}
  \label{fig:heatmap}
\end{figure}
The intensity of the red color indicates how often a particular token was visited.
A token highlighted by the lightest shade of red was visited twice while a token highlighted by the darkest shade of red was visited over 256 times.
This figure demonstrates several of the optimizations discussed in section~\ref{sec:optimizations}.
Firstly, thanks to the \texttt{dequeueOnNewStatements} optimization, the background is plain white up to the \texttt{Seq}.
The \texttt{Seq} gets visited twice, once when there's a space after the \texttt{=} and once when there's a newline.
Secondly, due to the OptimalToken optimization, when the search gets into trouble it backtracks to the tuple \texttt{(0, 0)} instead of the \texttt{Seq[((Int, Int), Matri)]} type signature.
Finally, because of the strategically placed comment at the end that exceeds the column limit, the search space grows out of bounds on the fourth argument triggering the \texttt{escapeInPathologicalCases} best-effort fallback.
Without heatmaps, it would be a much greater challenge to get these insights.
However, these heatmaps gave us limited insights in how our changes affected the search space in the best-first search.

We developed an extension to heatmaps that allows us to visually compare the difference in search space between two versions of scalafmt.
Figure~\ref{fig:heatmap2} shows an example of such a diff report.
\begin{figure}
  \centering
  \includegraphics[width=\textwidth]{img/heatmap2.png}
  \caption{Example diff heatmap}
  \label{fig:heatmap2}
\end{figure}
The green background indicates that the new version of scalafmt makes fewer visits to those regions.
Observe that the \texttt{>} operator has a background with a light shade of red.
This means that the operator was visisted more often in the new scalafmt version.
A price well worth paying considering the overall shrink in search space.
To produce diff heatmaps, we first persist to a database the statistics report needed to generate a single heatmap after each test run.
Then, we generate the diff heatmap by fetching two reports and calculating the difference in visits per token.
If the difference is negative for a particular token --- meaning we visited said token fewer times --- the background is highlighted green, otherwise red.

\subsection{Property based testing}~\label{sec:testing}
Property based tests verify that software implements a high-level specification.
Unlike example based tests, the input is not manually curated by the software developer.
Instead, the input is automatically generated.
Property based tests played a vital role in development of scalafmt and gave us confidence that the algorithms in section~\ref{sec:algorithms} behave well in the real world.
We collected a sample of 1.2 million lines of code from open source Scala projects available online.
The sample was compressed into a 23mb zip file\footnote{See \url{https://github.com/olafurpg/scalafmt/releases/download/v0.1.4/repos.tar.gz}}.
Our test suite would download the sample and test again three properties: \emph{can-format}, \emph{AST integrity} and \emph{idempotency}.

\subsubsection{Can-format}
The can-format property simply says that if the Scala compiler's parser is able to parse the source input file, then scalafmt should be able to format the source file.
Although this may seem like a trivial property, it was by far the most effective property at finding bugs in scalafmt.
Most commonly, comments in the most unexpected placed would cause the best-first search to not reach the last token in the input.
An overly strict Policy was usually the culprit of such bugs, which was easy to fix thanks to our tracing techniques described in section~\ref{sec:router}.

\subsubsection{AST integrity}
The AST integrity property says that the abstract syntax tree of the formatted source file should be identical to the abstract syntax tree of the original input.
Recall from section~\ref{sec:scalameta} that scala.meta trees can be serialized into strings.
We leverage this feature to test AST integrity.
Algorithm~\ref{alg:ast} shows the code needed to test AST integrity.
\begin{algorithm}
  \caption{AST integrity property}\label{alg:ast}
  \lstinputlisting[nolol]{code/ast.scala}
\end{algorithm}
This property catched several critical bugs.
For example, in one case, scalafmt inserted a newline after the keyword \texttt{return}, breaking the semantics of the original source code.
Moreover, this property highlighted the danger of enabling the \texttt{stripMargin} alignment.
Since the \texttt{stripMargin} modified the contents of regular and interpolated string literals,
the AST of the formatted output changed.
A sign that \texttt{stripMargin} should be disabled by default.
Finally, by adhereing to AST integrity, we confine the scope of what scalafmt can do.
This makes it easy for us to reject feature requests such as sorting imports.
Such refactoring can be left to a separate tool, in our humble opinion.
% For example, if scalafmt would sort imports it would change the AST.
% In Scala, imports can be relative and, hence, sorting them alphabetically can alter semantics.

\subsubsection{Idempotency}
The idempotency property says that if the output of formatting a source file twice should be identical to the output of formatting it once.
This property is critical for scalafmt to be used as part of any continuous integration setup.
It is not at all obvious that the algorithms in section~\ref{sec:algorithms} fulfill the idempotency property.
Our experience reveals that it is in fact very easy to accidentally introduce non-idempotent formatting rules in the Router.
We did not test for idempotency until the 0.2.3 release, after users reported non-idempotent formatting behavior in scalafmt.
Yet, even after we started testing against idempotency in our comprehensive test-suite, we continued to receive issues with non-idempotent formatting.
It turns out that 1.2 million lines of code is not a large enough sample to catch all property bugs.

% \subsection{Configuration}
% \subsubsection{maxColumn}
% \subsubsection{binPacking}
% \subsubsection{vertical alignment}
